\chapter{Conclusion}%    \chapter{}  = level 1, top level

	This research revealed significant findings regarding container dwell time prediction at the Ports of Auckland. It
	identified temporal patterns and specific characteristics related to the containers that influence dwell times.
	None of the weather-related variables, such as wind conditions, recorded by the port demonstrate significant
	predictive gain for the implemented models.

	The study established three meaningful dwell time categories that proved effective for predictive modeling:
	\begin{itemize}
		\item Short-term (0-3 days)
		\item Medium-term (4-11 days)
		\item Long-term (12+ days or more)
	\end{itemize}

	The proposed predictive models had a robust performance, with F1 Scores exceeding 73\%
	for most weeks in the analysis period 2024. Only one week showed a lower performance of 69\%
	in the F1 score, which is still a good metric for its predictive power. Among the machine learning
	algorithms evaluated, Logistic Regression emerged as the most effective approach, followed closely by
	K-Nearest Neighbors.


	These models can become critical components to be integrated into a Decision Support System (DSS) workflow to know
	how long a container will stay, which is essential in automating container allocation decisions. The Ports of
	Auckland could significantly enhance its operational efficiency by integrating them to address yard optimization
	challenges with data-driven solutions, which aligns with the primary goals of this project.

	These models are highly reliable for operation planning, given that their performance suggests their predictive
	power is significant. The following valuable steps are implementing them in a real-time DSS environment and
	evaluating their performance in terms of time, costs, and rehandles.


	\section{Future Work}

		This research has established a foundation for predicting container dwell times that could help improve yard
		optimization at the Ports of Auckland. From data movements, promising avenues for future research have been
		identified that could improve the effectiveness and applicability of this work.

		\subsection{Granular Yard-Level Analysis}

			The current models operate at a port-wide level. However, implementing a more granular yard-level analysis
			provides new research opportunities through the development of location-specific predictive models. These
			models would account for individual yard section characteristics, specific operational constraints of
			different areas, and local traffic patterns and accessibility. At this level, which is more granular, it
			could provide better decision-making and practical and specific yard optimization strategies.

		\subsection{Extended Training Data Implementation}

			The current models were trained with nine months of 2023 data. Expanding the training dataset could improve
			metrics by including additional months (January, February, and March). By adding these months, 2023 will be
			complete for container movements, capturing missing seasonal patterns at the beginning of the year and
			strengthening the model's understanding of monthly trends. This temporal expansion could improve the model
			and its performance for the testing dataset and its months of January, February, and March.

		\subsection{Weather Data Integration}

			Data augmentation through meteorological data could enhance prediction metrics by integrating key weather
			features, including precipitation patterns and visibility data. This would enable a comprehensive analysis
			of weather-related impacts on container handling efficiency and dwell time variations. These new features
			could provide valuable insights into environmental factors affecting port operations.

		\subsection{Model Simulation Framework}

			The evaluation of model-driven decisions in container yard operations presents significant challenges due
			to the complex nature of container movements. Testing the effectiveness of predictive models in reducing
			unnecessary movements is more complex than comparing simple before-and-after scenarios. Each container's
			movement can affect multiple others, creating a ripple effect throughout the yard that's difficult to track
			and quantify. For instance, moving one container might require shifting several others, and these
			interdependencies multiply rapidly across thousands of containers.
			\\
			\\
			Current data allows tracking actual movements, but simulating alternative scenarios - what could have
			happened if containers were placed differently based on predicted dwell times - remains challenging. This
			testing environment would need to copy actual port behavior: how putting a container in one spot limits
			where others can go, how knowing a container's expected stay time affects where it should be placed,
			and how all these choices add up over days and weeks of operation. Simulations would allow ports to
			experiment with different container management strategies, helping to prove whether these prediction models
			make ports run more smoothly and efficiently.

		\subsection{DSS Implementation}

			A practical next step would be building a user-friendly Decision Support System (DSS) that puts these
			prediction models to work in day-to-day port operations. This system would take real-time data about
			containers, run it through our models (mainly using Logistic Regression backed up by KNN), and give yard
			operators straightforward suggestions about where to place containers. The idea is simple but powerful - if
			it is known how long a container will likely stay, making decisions can be more thoughtful about where we
			put it. A container leaving soon should be easily accessible, while one staying longer can go to a less
			convenient spot. By tracking how many containers move, what we avoid, and how much faster we can get
			containers in and out, metrics could show exactly how much time and money these predictions save. This
			real-world testing would be the best way to prove whether these models can make ports run more efficiently.

		\subsection{More sofisticates classifiers}

			Adding new prediction models - like Neural Networks - could help spot patterns in container movements that
			simpler methods might miss. Adding a third model to work alongside the current two could create a voting
			system where the models work together to make better predictions.