\chapter*{Abstract}
	\setcounter{page}{1}
	\pagestyle{headings}
% \pagenumbering{roman}

	\addcontentsline{toc}{chapter}{Abstract}


	\begin{abstract}
		This research industry project presents a machine learning approach to predict container dwell times at the
		Ports of Auckland, aiming to optimize yard operations through data-driven decision support. The study analyzed
		over 1.5 million container movement records to develop predictive models that could effectively categorize
		container dwell times into three meaningful ranges: short-term (0-3 days), medium-term (4-11 days), and
		long-term (12+ days). Multiple machine learning algorithms were evaluated, including Random Forest, K-Nearest
		Neighbors, Logistic Regression, and Gradient Boosting. Logistic Regression was the most effective model,
		achieving F1 scores consistently above
		73\% in weekly predictions during the first quarter of 2024, with solid performance in short-term ( 86\% F1
		score) and medium-term (80\% F1 score) predictions. K-Nearest Neighbors demonstrated complementary capabilities
		as a secondary validation model. The research established that container characteristics and temporal
		patterns were significant predictors of dwell time, while environmental factors such as wind conditions
		recorded by Ports of Auckland, provided for training, showed no meaningful impact on prediction accuracy.
		Models' strong performance in short- and medium-term scenarios, represented in most operational cases,
		indicated their readiness for practical implementation. Both models struggled with long-term predictions.
		This study also proposed integrating these predictive models into a Decision Support System (DSS) to automate
		container allocation decisions to reduce unnecessary container movements and improve space utilization. The
		findings provide bases for enhancing operational efficiency in port yard management, with clear pathways for
		future development through extended data collection, granular yard-level analysis, and real-time DSS
		implementation.
	\end{abstract}