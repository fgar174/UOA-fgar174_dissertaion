\section{Interpretation}
	In the analysis, the models that stood out due to their commendable accuracy were logistic regression (75.6\%)
	(Figure ~\ref{fig:log_reg}), random forest(Figure ~\ref{fig:rand_fore}) (79.34\%), and gradient boosting (85.41\%)
	(Figure ~\ref{fig:grad_boos}). These models consistently delivered reliable results, making them optimal choices
	for the objectives of this study. The precision of these models in predicting outcomes highlighted their
	efficiency and relevance for this particular data set and research goal.

	\subsection{Logistic Regression}

		\begin{figure}[h!]
			\centering
			\includegraphics[scale=1]{images/dm_confu_mat_logi_regr}
			\caption{Confusion Matrix Logistic Regression}
			\label{fig:log_reg}
		\end{figure}
		\begin{table}[h!]
			\centering
			\caption{Feature Importance Logistic Regression}
			\begin{tabularx}{0.8\textwidth}{lX}
				\toprule
				\textbf{Feature}                   & \textbf{Importance} \\
				\midrule
				social\_support                        & 4.0258              \\
				positive\_affect                       & 0.54279             \\
				life\_ladder                           & 0.38305             \\
				perceptions\_of\_corruption            & 0.15111             \\
				pig                                    & 0.00973             \\
				fish\_and\_seafood                     & 0.00955             \\
				beef                                   & 0.00204             \\
				prevalence smoking                     & -0.00189            \\
				liters\_of\_pure\_alcohol\_per\_capita & -0.01522            \\
				sheep\_and\_goat                       & 0.03352             \\
				poultry                                & -0.03421            \\
				generosity                             & -0.17302            \\
				freedom\_to\_make\_life\_choices       & 0.45903             \\
				negative\_affect                       & 0.49376             \\
				\bottomrule
				\label{table:fi_logistic}
			\end{tabularx}
		\end{table}
		\\
		\\
		In the logistic regression results, social\_support has an importance of 4.0258, which is notably higher than
		any other feature. This implies that a strong community, interpersonal relationships, and the overall feeling
		of being supported can significantly influence the dependent variable, presumably related to health outcomes.
		Following social support, we have other indicators of emotional well-being, such as positive\_affect with the
		importance of 0.54279 and negative\_affect with 0.49376. These values suggest that the presence of positive
		emotions and the absence of negative ones are also vital contributors. Moreover, life\_ladder, which indicates
		overall life satisfaction, has an importance of 0.38305, reinforcing the idea that mental well-being and life
		satisfaction play crucial roles.
		\\
		\\
		In comparison, dietary and lifestyle factors like pig, fish\_and\_seafood, beef, and
		liters\_of\_pure\_alcohol\_per\_capita have much lower importance values, highlighting that while they might
		play a role, they are not as influential as the aforementioned mental and emotional well-being factors.
		Thus, when interpreting the results of this logistic regression, it's clear that mental and emotional health
		attributes overshadow other factors in their impact. This emphasizes the importance of mental health and its
		intricate link to broader health outcomes.
		\\
		\\

		\paragraph{Insights:}
			The main insight from the model logistic regression based in its feature importance
			~\ref{table:fi_logistic}:
			\\
			\\
			Social Support: The strongest predictor, social support, suggests a prominent role in maintaining lower
			obesity rates. Nations with stronger community ties and support networks might be better equipped to combat
			obesity.
			\\
			\\
			Emotional Well-being: Both positive affect and negative affect play a role in obesity rates. Regions with
			higher levels of happiness, contentment, and fewer negative emotions tend to have lower obesity prevalence.
			\\
			\\
			Life Satisfaction: Countries with a higher "life ladder" score, indicating overall life satisfaction,
			show a correlation with lower obesity rates.
			\\
			\\

		\paragraph{Possible Actions:}
			Holistic Health Education: Educational campaigns shouldn't just focus on diet but also on the importance
			of emotional well-being, life satisfaction, and community support.
			\\
			\\
			Empower Individual Choices: Recognize the power of individual autonomy and encourage informed
			decision-making about personal health and lifestyle.
			\\
			\\
			Foster Community Engagement: Governments and local bodies should promote community events, physical
			activity groups, and support networks that encourage healthy lifestyles.

\subsection{Random Forest and Gradient Boosting}

	\paragraph{Logistic Regression:}
		\begin{figure}[h!]
			\centering
			\includegraphics[scale=1]{images/dm_confu_mat_rand_fore}
			\caption{Confusion Matrix Random Forest}
			\label{fig:rand_fore}
		\end{figure}

		\begin{table}[h!]
			\centering
			\caption{Feature Importances Random Forest}
			\begin{tabularx}{0.8\textwidth}{lX}
				\toprule
				\textbf{Feature}                   & \textbf{Importance} \\
				\midrule
				pig                                    & 0.1423              \\
				poultry                                & 0.132               \\
				liters\_of\_pure\_alcohol\_per\_capita & 0.1029              \\
				life\_ladder                           & 0.0971              \\
				negative\_affect                       & 0.0909              \\
				beef                                   & 0.082               \\
				sheep\_and\_goat                       & 0.0725              \\
				fish\_and\_seafood                     & 0.0661              \\
				social support                         & 0.0552              \\
				generosity                             & 0.0445              \\
				prevalence\_smoking                    & 0.0357              \\
				perceptions\_of\_corruption            & 0.0279              \\
				positive\_affect                       & 0.0262              \\
				freedom\_to\_make\_life\_choices       & 0.0248              \\
				\bottomrule
				\label{table:fi_random}
			\end{tabularx}
		\end{table}

		\textbf{Mental Health Emphasis:} The most significant finding from the logistic regression model is the
		emphasis on mental and emotional health indicators. Factors like \texttt{social\_support},
		\texttt{positive\_affect}, and \texttt{negative\_affect} have high importance scores. This indicates that in
		the logistic regression model, mental well-being plays a central role in the prediction. The model essentially
		suggests that having a robust social support system and maintaining emotional equilibrium are primary
		determinants of the outcome.

	\paragraph{Random Forest}
		\\
		\\

		\subparagraph{Dietary Importance:} Unlike the logistic regression, the Random Forest model places a higher
			emphasis on dietary factors. Consumption patterns of pig, poultry, beef, and other meats suggest that
			diet is
			a pivotal component in determining outcomes when using this model.

		\subparagraph{Moderate Mental Health Emphasis:} Although dietary patterns dominate, aspects of mental and
			emotional
			health like life satisfaction (\texttt{life\_ladder}) and negative emotions (\texttt{negative\_affect})
			still
			find relevance, but they are overshadowed by the diet.

		\subparagraph{Additional Factors:} Alcohol consumption is also highlighted, pointing to its significance in
			influencing the outcome. While social support is still relevant, its importance is notably reduced
			compared to
			the logistic regression.
			\\
			\\

	\paragraph{Gradient Boosting}
		\\
		\\

		\subparagraph{Dietary Dominance:} Similar to the Random Forest (Table ~\ref{table:fi_random}), Gradient
			Boosting
			(Table ~\ref{table:fi_gradient}) also underscores the importance of dietary patterns, with poultry and pig
			taking the lead. This consistency between the two models reaffirms the significance of diet.

		\subparagraph{Balanced Emphasis on Emotional Health:} While diet is primary, there's still an emphasis on the
			balance between negative and positive emotions, suggesting that emotional well-being continues to play a
			role,
			albeit lesser than in logistic regression.

		\subparagraph{Varied Social Factors:} The importance of social aspects, such as social support, has further
			decreased. However, the model sheds light on other societal factors like perceptions of corruption and the
			freedom to make life choices.

			\begin{figure}[h!]
				\centering
				\includegraphics[scale=1]{images/dm_confu_mat_grad_boos}
				\caption{Confusion Matrix Gradient Boosting}
				\label{fig:grad_boos}
			\end{figure}
			\begin{table}[h!]
				\centering
				\caption{Feature Importances Gradient Boosting}
				\begin{tabularx}{0.8\textwidth}{lX}
					\toprule
					\textbf{Feature}                  & \textbf{Importance} \\
					\midrule
					poultry                                & 0.1528              \\
					pig                                    & 0.1069              \\
					liters\_of\_pure\_alcohol\_per\_capita & 0.1045              \\
					sheep\_and\_goat                       & 0.1023              \\
					fish\_and\_seafood                     & 0.0926              \\
					life\_ladder                           & 0.0907              \\
					beef                                   & 0.068               \\
					negative\_affect                       & 0.0509              \\
					positive\_affect                       & 0.0507              \\
					prevalence\_smoking                    & 0.0447              \\
					perceptions\_of\_corruption            & 0.0387              \\
					social\_support                        & 0.0373              \\
					generosity                             & 0.0348              \\
					freedom\_to\_make\_life\_choices       & 0.025               \\
					\bottomrule
					\label{table:fi_gradient}
				\end{tabularx}
			\end{table}
			\\
			\\
			The analysis of feature importance across Logistic Regression, Random Forest, and Gradient Boosting models
			underscores the multidimensional nature of the factors influencing the outcome. The Logistic Regression
			model
			prioritizes mental and emotional health, emphasizing the centrality of social support and emotional
			balance in
			predictions. In contrast, both Random Forest and Gradient Boosting pivot towards dietary patterns,
			highlighting
			their pivotal role while still acknowledging the role of mental and emotional health, albeit to a lower
			extent.
			Furthermore, the subtle differences in feature importance, such as the emphasis on alcohol consumption in
			Random Forest and societal factors in Gradient Boosting, serve as a testament to the diverse perspectives
			these models bring to the table. Finally, while each model offers unique insights, together, they provide a
			comprehensive understanding, suggesting that both dietary habits and mental well-being, along with societal
			influences, are integral to the outcome in question but do not support each other conclusion.
