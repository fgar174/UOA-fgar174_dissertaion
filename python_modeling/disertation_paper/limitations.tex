\chapter{Limitations}

	Considering its findings and possible applications, some limitations that should be acknowledged are related to
	this dissertation.


	\section{Stakeholder Information Limitations}

		The model training could be improved by adding more features related to operators or companies handling
		containers. Information regarding container origins and destinations or details about transportation companies
		were unavailable. This lack of stakeholder-specific data restricts finding patterns related to operators and
		company-specific operational tendencies.


	\section{Cargo Content Information}

		The research did not use cargo information, presenting another notable limitation. With no access to data
		related to container contents and cargo priority, the training process could not learn from that information.
		This limitation affects the understanding of content-specific dwell time patterns, prevents analysis of
		differences between high-priority and standard cargo handling, and limits the identification of content-based
		operational requirements that might influence container dwell times.


	\section{Environmental Data Integration}

		Integrating external official datasets to capture weather effects in port operations was challenging. The
		weather dataset for 2023 was not available at the moment this project was performed, making it hard to see how
		weather conditions might influence container movements. While the port provided some weather information, like
		the recorded wind data, this did not significantly affect how well the models worked. Adding additional
		features related to weather conditions could allow the models to predict container dwell times better. Future
		research could look into how heavy rain or extreme temperatures affect dwell times by using official data to
		make predictions even more accurate.


	\section{Temporal Data Constraints}

		The training data was limited to nine months, from the middle of March to December 2023, while testing data
		covered only January through March 2024. Without a complete annual cycle in the training data, this restriction
		could impact the model's capabilities to identify annual patterns in container movements for long-term
		trends in port operations to capture seasonal variations that might occur throughout the year.

		\subsection{Time Series Analysis}

			Having just one year of data (2023-2024) limits the use of powerful prediction tools that work with
			patterns over time. These time series models need several years of data to spot reliable patterns like
			seasonal changes or yearly trends in how containers move through the port. For example, the training and
			analysis could not see how summer peaks compare to winter lows across different years or how holiday
			seasons affect container stay year after year.
			\\
			\\
			While the current models work well for range predictions, having more years of historical data would let us
			use specialized tools with different approaches to generate insights for longer-term patterns. A time
			series analysis could help the Ports of Auckland better understand busy seasons and quiet periods and
			create new features for training. It is something worth looking into once more historical data becomes
			available.


	\section{Scheduling Information Gaps}

		The absence of scheduled pickup information presented another significant constraint in this research. It is
		worth mentioning that was why predictions were necessary.