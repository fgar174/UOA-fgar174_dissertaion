\chapter{Literature Review / Related Works}%    \chapter{}  = level 1, top level
	Container stacking and decision support systems (DSS) have evolved significantly in port operations management.
	\\
	\\
	While numerous studies have addressed container stacking problems
	\cite{gharehgozli2014decision, ries2014fuzzy, bazzazi2009genetic, guven2019modelling, guerra2018heuristic,
		nishimura2009container, sharma2014genetic, park2011dynamic, goerigk2016robust}, developing comprehensive DSS
	solutions has been relatively limited. Notable exceptions include the work of Murty et al.
	\cite{murty2005decision}, Liu et al. \cite{liu2010decision}, and Legato and Mazza \cite{legato2018decision},
	though these studies did not integrate predictive analytics or consider dwell times in their optimization models.
	\\
	\\
	Even though dwell time performance is an important indicator, affected by multiple factor (Table~
	\ref{tab:dwell_time_factors}
	), a gap exists in the literature regarding container dwell time prediction to solve optimization yard
	problems. Only some studies address this challenge; Moini et al. \cite{moini2012estimating}
	analyzed U.S. ports, identifying crucial features such as port type, geographical location, and container traffic
	patterns. Based on this foundation, Kourounioti et al. \cite{kourounioti2016development}
	employed regression models and neural networks to predict dwell time.
	\\
	\\
	The hierarchical approach to decision-making has emerged as a common framework in container terminal
	operations.
	As documented by several researchers \cite{zhang2003storage, chen2012storage, park2011dynamic}, this two-stage
	process first considers block assignment at an aggregated level, followed by real-time decisions about specific
	container locations. Some insights about storage yard operations are provided by Carlo et al.
	\cite{carlo2014storage}.
	\\
	\\
	Historical perspectives reveal the evolution of DSS applications for container handling. Starting with the
	early contribution from Van Hee and Wijbrands \cite{van1988decision}, the field has expanded to cover many
	operational aspects.
	Recent studies have addressed specific challenges such as real-time transportation planning
	\cite{bandeira2009dss, van2016real, shen1995dss}, crane scheduling, berth allocation
	\cite{wang2007stochastic}, and the integrated berth allocation and quay crane assignment Problem (BACAP) problem
	\cite{ursavas2014decision}.
	\\
	\\
	The effectiveness of category-based stacking policies in reducing rehandles has been well-documented. As a
	reference, Dekker et al. \cite{dekker2007advanced} demonstrated the benefits of category-based stacking. On the
	other hand, Borgman et al. \cite{borgman2010online} have highlighted the importance of departure time-based
	classification. Historical data can reduce operational costs, improving service levels for port users by
	leveraging and designing more effective stacking rules.
	\\
	\\
	This research builds upon a study by Gaete et al. \cite{gaete2018novel}, which explored dwell time prediction using
	multi-class classification. Also, follows some of the ideas for a DSS architecture proposed by Maldonado et al.
	\cite{maldonado2019analytics} which uses machine learning models for decision-making in port operations across a
	decision support system.
	\\
	\\
	The current approach introduces a methodology for dwell time prediction by classifiers adjusted to The Ports of
	Auckland's operational context. It explains how these predictions could be integrated into DSS architecture that
	could be implemented in real-world port operations to optimize the yard finally.

	\begin{table}[ht]
		\centering
		\begin{tabular}{|p{0.45\textwidth}|c|c|}
			\toprule
			\hline
			\textbf{Factor} & \textbf{References}
			& \textbf{Type} \\
			\hline
			\midrule
			Vessel sailing schedule frequency & \cite{merckx2005issue, moini2012estimating}
			& Unique value
			\\
			\hline
			Container specifications\textsuperscript{*} & \cite{merckx2005issue, moini2012estimating}
			& Nominal
			\\
			\hline
			Hinterland connections & \cite{merckx2005issue, moini2012estimating}
			& Unique value \\
			\hline
			Port governance structure & \cite{merckx2005issue, moini2012estimating}
			& Unique value \\
			\hline
			Terminal location \& logistics & \cite{merckx2005issue, moini2012estimating}
			& Unique value
			\\
			\hline
			Terminal operations schedule &
			\cite{merckx2005issue, moini2012estimating, rodrigue2009terminalization}
			& Unique value
			\\
			\hline
			Shippers and consignees & \cite{moini2012estimating, rodrigue2009terminalization}
			& Nominal
			\\
			\hline
			Regulatory procedures & \cite{moini2012estimating}
			& Unique value \\
			\hline
			Transport corridors & \cite{moini2012estimating}
			& Nominal \\
			\hline
			Maritime shipping details\textsuperscript{†} & \cite{moini2012estimating}
			& Nominal \\
			\hline
			Container flow balance & \cite{moini2012estimating}
			& Nominal \\
			\hline
			\bottomrule
			\multicolumn{3}{l}
			{\footnotesize \textsuperscript{*}Including type (empty/full, dry/reefer), size (20/40 TEUs), contents}
			\\
			\multicolumn{3}{l}{\footnotesize \textsuperscript{†}Ocean carriers and empty container demurrage time}
			\\
		\end{tabular}
		\caption{Factors Influencing Container Dwell Time \cite{gaete2018novel}}
		\label{tab:dwell_time_factors}
	\end{table}